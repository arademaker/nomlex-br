\documentclass{article}
\usepackage{url}
\usepackage{times}
\usepackage{latexsym}
\usepackage{amsfonts,amssymb,amsmath}
\usepackage{enumerate}
\usepackage{nameref}

\newcommand{\logic}[1]{\textsf{#1}}
\newcommand{\abrev}[1]{\textsf{#1}}

% correct bad hyphenation here
\hyphenation{op-tical net-works semi-conduc-tor}

\begin{document}

\title{NOMLEX-BR: \\ A Five Finger Exercise in Lexical Resource Creation}
\author{Valeria de Paiva \and Alexandre Rademaker}
\maketitle

\begin{abstract} 
  We describe a Brazilian Portuguese version of the original English
  NOMLEX lexical resource created by the project
  Proteus~\cite{nomlex98}. We first describe NOMLEX, then we explain
  why we believe a Brazilian Portuguese version (which we call
  NOMLEX-BR), should be useful, in the context of a recently started
  project on Knowledge Representation in the FGV. Then we outline some
  of the other, more substantial work that we plan to engage in for
  the project.
\end{abstract}


\section{Introduction}

% We study the semantics of nominalizations, and its implications for
% Natural Language Processing applications such as electronic
% ontologies or Information Retrieval.

Our aim is to discuss the production and distribution of an
electronic, small, lexicon of nominalizations in Brazilian Portuguese,
as well as a semantically annotated corpus of examples of these
deverbal nouns.

We focus on nominalizations in this work, for several reasons.
Deverbal nouns, or nominalizations, can pose serious challenges for
knowledge-representation systems. A sentence like ``Alexander
destroyed the city in 332 BC'' can be easily parsed and its semantic
arguments, such as the agent of destruction (Alexander), the thing
destroyed (the city) and the time (332 BC), are readily obtained for a
proposed logical representation of the sentence. By contrast, a
sentence like ``Alexander's destruction of the city happened in 332
BC'' is much harder to deal with. It describes the same event of
destruction, with the same semantic arguments, but these arguments are
harder to obtain automatically from a syntactic parsing of the
sentence.

Nominalizations are well-studied in English, with the NOMLEX project
(\cite{nomlex98}) providing an well-established, open access baseline
for corresponding results in other languages. Our work in NOMLEX-BR
builds up from previous work on nominalizations in English
\cite{flairs2006}. This work extends the coverage of NOMLEX English
norminalizations, via the use of Xerox PARC's state-of-the-art natural
language processing system XLE \cite{xle} and some simple, but
effective heuristics and compared it to NOMLEX-PLUS~\cite{nomlexplus},
the state-of-the-art in 2004. Our work is here is an attempt at
building the basic blocks underlying that, for Brazilian Portuguese.
We hope that the work done for English can be suitably adapted and
re-used for Portuguese, if we keep the languages comparisons in place.
We also hope to learn and adapt from the French experience with
nominalizations, described in the Nomage project~\cite{frenchnomlex}.

The original version of NOMLEX, is a small resource, only around a
thousand nominalizations, which seemed ideal to kick off a new
collaborative project between the investigators, who are trying to
work together in a field (lexical resource creation) that turns out to
be new to both.

The original NOMLEX was constructed starting out with nominalizations
with the -ion, -ment and -er suffixes, taking samples of the most
frequent words first in a list of nouns from a combination of the
Brown Corpus and the Wall Street Journal (about 1 million words of
each). Words with these kinds of suffix tend to be erudite words and
these tend to work similarly in different (but related) languages, was
the working hypothesis, which seems confirmed, to some degree by our
(admittedly very small) prototype.

\section{NOMLEX}\label{nomlex}

NOMLEX is a lexicon of English nominalizations developed by the group
at New York University for many years. It relates the arguments of a
nominalization to the predicate argument structure of its associated
verb, but it does not require exactly the same structure for the
nominal and the verbal lexical item. It also records details of the
syntactic realization of the arguments, including prepositions
associate with the arguments. For example, the entry for "promotion"
in NOMLEX reads:

\begin{verbatim}
(nom :orth promotion
     :verb promote
     :nom-type((verb-nom)) 
     :verb-subj ((n-n-mod) (det-poss))
     :verb-subc ((nom-np :object ((det-poss)(n-n-mod)(pp-of)))
		 (nom-np-as-np :object ((det-poss) (pp-of)))
		 (nom-possing :nom-subc ((p-possing :pval (of))))
		 (nom-np-pp :object ((det-poss) (n-n-mod) (pp-of))
			    :pval (into from for to))
		 (nom-np-pp-pp:object ((det-poss) (n-n-mod) (pp-of))
				      :pval (for into to) :pval2 (from))))
\end{verbatim}

Our Brazilian Portuguese version keeps the original structures of the
original English version of NOMLEX, but adds an extra field
corresponding to some usage example in Portuguese. This is usually
called a `gloss' in WordNet\cite{wordnet}. Glosses for NOMLEX-BR were
obtained using the `Corpus do Portugu\^es' \cite{corpusdoportugues}.

%\section{Related work}

\section{Why NOMLEX-BR?}\label{nomlex-br}

We recently started a project called ``Logics and Ontologies for
Brazilian Portuguese'' whose ultimate aim is to represent, in a
suitably described logic, the meanings of sentences in Brazilian
Portuguese. Given that this work is to be conduced at distance (one of
us is in California, the other in Brazil) and given that this is
somewhat a labor of love and not many resources are allocated to it,
it makes sense to build up our systems in smaller chunks.

Lexical resources for languages other than English are notoriously
difficult to come by. The fact that there is not even a version of a
Portuguese WordNet freely available for download and modification by
anyone is a clear indication of the difficulties ahead. To follow
somewhat the traditional pipeline for logic based systems based on
language, e.g. the one described by the Bridge system of PARC
(\cite{bridge}) we need a collection of lexical resources as well as
(much as possible) off-the-shelf systems.

Ideally we would want to have a broad coverage, deep processing LFG
grammar of Brazilian Portuguese and while we are pursing leads in this
direction (\cite{leonel04}), this may take a while to get, as
hand-crafted large coverage grammars are very labor intensive.
Meanwhile we thought we would experiment with off the box Portuguese
parsing in the style of the Stanford parser, adapted by the the group
at the University of Lisbon, led by Prof Antonio Branco
(\cite{off-the-box}). At the same time, it seems sensible to construct
ourselves some of the resources that we are more familiar with, and a
small version of NOMLEX for Portuguese, NOMLEX-BR seems just the ideal
small project to get things rolling.

Another somehow indirect route that we are taking towards our goal is
to consider a generic, open source ontology like
SUMO/Sigma~\cite{sumo} and trying to adapt it to Portuguese concepts.

\section{Knowledge Representation and Textual Inference}\label{til}

% The use of deverbals in traditional Knowledge Representation tends
% to be obscured by the active role played by the ontologists. They
% usually decide if a certain concept will be represented as a process
% (mostly using gerunds such as {\sf Combining}) or if one should use
% a noun. Most knowledge representation languages will use nouns for
% all concepts that one needs to deal with. But clearly verbs, which
% tend to represent processes or events are as important as nouns, if
% not more.

Knowledge representation languages tend to be based on concepts,
usually denoted by nouns. Verbs, which tend to represent processes or
events, are as important as nouns, if not more so, but somehow it is
less clear how to deal with them in traditional logic-based knowledge
representation. The ontologists or model constructors normally decide
which concepts they consider the most productive ones for a given
domain. They also decide how to represent the most productive concepts
via a judicious mixture of processes and events. Eventually whether
using first-order or higher-order logic or description logic or modal
logics or a combination of all the above, one ends up with a
collection of predicates that seem very {\em ad hoc}.

Given that there are many different frameworks for knowledge
representation and that the reasons for choosing particular features
of frameworks are very varied too, it is a hard task to compare
frameworks. To decide whether a representation in logical form of a
sentence in one framework is better than another is also a difficult
task. Of course once a whole theory has been formalized in a given
framework as a mathematical object, one can do the usual things that
one does with a logic, compare it to other logics, prove the usual
traditional theorems, etc. But measuring the {\em adequacy} of your
logical formalization when compared to the raw phenomena you started
with is a hard job. If the phenomena you've started from is described
in a collection of sentences, maybe it is easier to consider some
notion of {\em textual inference} in the original collection of
sentences.

Textual inference is an informal relationship between two pieces of
text where the first text (the premiss $P$) is supposed to entail the
second text (the hypothesis $H$).That is, the second text $H$ follows
\emph{logically}, but not necessarily formally from te the first text
(the premise $P$). Once the content of the texts has been formally
rendered as pieces of logic, we expect the corresponding logical
expressions to entail in the same direction $P\to H$.

The same way using textual entailment bypasses some of the problems of
deciding whether a particular rendering in logic of a sentence is or
is not adequate, the use of large scale knowledge representation
systems together with textual entailment tasks should help us decide
which predicates are the most useful ones. The comparison between
deverbal notions of concepts and their corresponding verbal versions
also should help with the required choice between predicates.
 

\section{Further Work}

This small lexicon of deverbals is just a first step. Certainly it
would be useful to have similar versions of nominalizations of
adjectives and adverbs, which also need a common concept mapping. Also
in the immediate plans is a version of VerbNet-BR, as the syntactic
alternances captured by the original VerbNet~\cite{verbnet} correspond
to useful semantical information. Again there is some hope that some
of the original Levin classes used for the construction of VerbNet are
also valid in Portuguese, but this is mostly a hope, so far.

Summarizing the creation of linguistic resources requires openness of
programs and of code. The only way to keep alive some/any resource is
make sure that people can modify it to their own purposes, be they
commercial or not. If one wants the enterprise of automatic language
understanding to flourish one must make sure that lexical resources
exist, are freely available and easy to use.

\bibliographystyle{plain}
\bibliography{nomlex-br}

\end{document}
 	
